\errorcontextlines9999
\documentclass[
	aspectratio=169, 
	10pt 
]{beamer}

\usepackage{fancybeamer} 
\usepackage{fancycolumn} 
\usepackage[dvipsnames]{xcolor}
\usepackage{colortbl}
\usepackage{array}

\newcommand\conly[2]{\only<#1>{#2}}

\title{Haskell Done Quick (02)} 
\author{Lars Pfrenger} 
\date{22. November 2024} 


\begin{document}

\maketitle

\section{Typen und Typklassen}

% 1. Warum ist ein typ system sinvoll?
%	1. Beispiel: Reddit Post Lua Skript
\subsection{Aber warum?}
\begin{frame}[t,fragile]{\insertsubsection}
	\large\href{https://www.reddit.com/r/programminghorror/comments/1gvkile/i_spent_3_hours_trying_to_find_the_bug_here/}{\faReddit \space "I spent 3 hours trying to find the 'bug' here"}\normalsize
	\bigskip
	\pause
	\setminted{escapeinside=||}
	\begin{minted}[autogobble]{lua}
		elseif args[1] == "SaveToolSlot" then
			local backpackStuff = Leaderstats:Get(player, "Backpack")
		
			for _,v in pairs(backpackStuff) do
				local |\conly{2}{match}\conly{3}{\colorbox{Red!50}{match}}| = string.match(v, args[2])
			
				if |\conly{2}{math}\conly{3}{\colorbox{Red!50}{math}}| ~= nil then
					backpackStuff[v] = args[2].."_"..args[3]
				end
			end
	
	\end{minted}
\end{frame}


%	2. Beispiel: Flugzeug
%	3. LiquidHaskell
\begin{frame}{\insertsubsection}
	\begin{itemize}
		\item Fehler erkennen bevor das Programm ausgeführt wird 
		\item Kritische Systeme (z.B. Flugzeuge)
		\item \href{https://ucsd-progsys.github.io/liquidhaskell/}{LiquidHaskell} $\rightarrow$ Beweise auf/mit Code führen
	\end{itemize}
\end{frame}

% 3. Primitive Typen 
\subsection{Primitive Typen}
\begin{frame}{\insertsubsection}
	\renewcommand{\arraystretch}{1.2}
	\begin{tabular}{>{\bfseries\color{red}}lll}
		Type & Wertebereich & Beispiel\\\\
		Bool & \mintinline{haskell}{False}, \mintinline{haskell}{True} & \mintinline{haskell}{False} \\
		Char & Zeichen & \mintinline{haskell}{'a'} \\
		String & Zeichenkette & \mintinline{haskell}{"Hallo Welt"} \\
		Int & Ganzzahlen mind. $[-2^{29}, 2^{29}-1]$ & $42$ \\
		Integer & Ganzzahlen (theoretisch) beliebig Groß & $2^{100}$ \\
		Float & IEEE single precision Gleitkommazahl  & 1.2345678 \\
		Double & IEEE double precision Gleitkommazahl & 1.234567890234567\\
	\end{tabular}
\end{frame}


% 4. Typ Klassen
%	1. Welches Problem lösen Typ Klassen?
\subsection{Warum Typ Klassen?}
\begin{frame}[fragile]{\insertsubsection}
	\begin{minted}[autogobble,escapeinside=||]{haskell}
		(==) :: |\conly{1}{Eq => }|a -> a -> Bool
	\end{minted}

	\onslide<3->{
	\begin{enumerate}
		\item[$\Rightarrow$] \mintinline{haskell}{(==)} würde für alle Typen valide sein 
		\item[$\Rightarrow$] man soll aber nur bestimmte Typen vergleichen können 
		\item[$\Rightarrow$] z.B. Funktionen vergleichen macht kein Sinn
	\end{enumerate}
	}
\end{frame}

%	2. Klassen Definition
\subsection{Typklassen Definition}
\begin{frame}[fragile]{\insertsubsection}
	
	Klassendefinition:
	\begin{minted}[autogobble,escapeinside=||]{haskell}
	class Eq |\conly{3}{\colorbox{Blue!40}}{a}| where 
	 |\conly{3}{\colorbox{Green!40}}{(==)}| :: a -> a -> Bool
	\end{minted}

	\pause
	\bigskip
	\bigskip

%	3. Instanz 
	Instanz:
	\begin{minted}[autogobble,escapeinside=||]{haskell}
	instance Eq |\conly{3}{\colorbox{Blue!40}}{Float}| where
	  |\conly{3}{\colorbox{Green!40}}{x == y}| = x `floatEq` y	
	\end{minted}

	\bigskip
	$\Rightarrow$ \mintinline{haskell}{(==)} kann auf \mintinline{haskell}{Float} benutzt werden 
\end{frame}


% 5. Syntax
	% 1. 5 :: Integer (wert mit typ assoziieren (read as "has type"))
	%  (==)                    :: (Eq a) => a -> a -> Bool  (contex / constraint)
\subsection{Syntax}
\begin{frame}[fragile]{\insertsubsection}
	\begin{minted}[autogobble,escapeinside=||]{haskell}
		(==) :: (Eq a) => a -> a -> Bool
	\end{minted}
	
	\vspace{-1.5em}
	\hspace{-0.1cm}
	$\underbrace{\hspace{0.5cm}}_{\text{Name}}$

	\vspace{-2.3em}
	\hspace{1.3cm}
	$\underbrace{\hspace{1.0cm}}_{\text{Constraint}}$

	\vspace{-2.3em}
	\hspace{3.2cm}
	$\underbrace{\hspace{2.6cm}}_{\text{Typsignatur}}$

	\pause
	\bigskip
	\bigskip

	$\Rightarrow$ "Wenn a die Klasse \mintinline{haskell}{Eq} implementiert, dann hat \mintinline{haskell}{(==)} den Typen \mintinline{haskell}{a -> a -> Bool}"

\end{frame}

\subsection{Typklassen}
\begin{frame}{\insertsubsection}
	% Florian, please forgive me for my sins 
	\includegraphics[width=\textwidth]{assets/typeklassen.png}
\end{frame}
% 6. toEnum fromEnum toIntegral

% 7. GHCI :t, :info

\subsection{GHCI}
\begin{frame}[fragile]{\insertsubsection}
	\begin{minted}[autogobble,escapeinside=||]{haskell}
		ghci> :info Num
		|\pause|
		type Num :: * -> Constraint
		class Num a where
		  (+) :: a -> a -> a
		  (-) :: a -> a -> a
		  (*) :: a -> a -> a
		
		  ...
		 
		instance Num Double -- Defined in ‘GHC.Float’
		instance Num Float -- Defined in ‘GHC.Float’
		instance Num Int -- Defined in ‘GHC.Num’
		instance Num Integer -- Defined in ‘GHC.Num’
		instance Num Word -- Defined in ‘GHC.Num’
	\end{minted}
\end{frame}



\begin{frame}{Ende}
	\center Danke für Eure Aufmerksamkeit
\end{frame}

\end{document}
