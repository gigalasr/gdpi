\documentclass[
	aspectratio=169, 
	10pt 
]{beamer}

\usepackage[utf8]{inputenc} 
\usepackage{fancybeamer} 
\usepackage{fancycolumn} 
\usepackage{listings}
\usepackage{graphicx}
\usepackage{tikz}
\usepackage{tcolorbox}
\usepackage{xcolor}
\usepackage{fontawesome5}
\usepackage{minted}
\usepackage{tcolorbox}
\usepackage{varwidth}

\title{Haskell Done Quick (01)} 
\author{Lars Pfrenger} 
\date{08. November 2024} 


\begin{document}

\maketitle

\section{Organisatorisches}
\begin{frame}{Who dis?}
	\faAt \mbox{} Lars Pfrenger\break 
	\faEnvelope \mbox{} lars.pfrenger@uni-ulm.de\break
	\faGraduationCap \mbox{} 5. Semester Bachelor Software Engineering 
	
	\begin{flushright}
        \includegraphics[width=0.3\textwidth, trim={0 0 12.5cm 0}]{assets/hello.png} % The image takes full width of this minipage
    \end{flushright}

\end{frame}


\begin{frame}{Übung}
	\begin{itemize}
		\item 50\,\% auf 5/7 Blättern zum Bestehen
		\item Blätter sind eine eigene Prüfungsleistung!
	\end{itemize}
\end{frame}

\begin{frame}[fragile]{Tutorium}
	\begin{fancycolumns}[T,widths={60,40}]
		\begin{itemize}
			\item Jede Woche in 027/122
			\item Neues Blatt jede 2. Woche
			\item Abgabe bitte nicht handschriftlich \break (.pdf, .txt, .hs)
			\item \underline{Nur Code der kompiliert wird bewertet} \begin{enumerate}[$\rightarrow$]
				\item Nicht funktionierenden Code auskommentieren
			\end{enumerate}
		\end{itemize}
		\nextcolumn
		\begin{minted}[autogobble]{haskell}
			-- Einzelner Kommentar

			{- 
				Kommentar der
				über mehrere  
				Zeilen geht :)
			-}
		\end{minted}
	\end{fancycolumns}
\end{frame}

\begin{frame}{Tutorium}
	\begin{description}[Woche A:]
		\item[\textbf{Woche A}:] Lösungen Besprechen + Infos fürs nächste Blatt
		\item[\textbf{Woche B}:] Infos fürs Blatt + Zeit zum Blatt bearbeiten + Fragen stellen
	\end{description}
\end{frame}

\begin{frame}{Tutorium}
	\center Stellt Fragen :D
	\center Themenswünsche oder Fragen gerne per Mail an lars.pfrenger@uni-ulm.de
\end{frame}

\section{Haskell}
\begin{frame}{Haskell}
	\begin{fancycolumns}
		\begin{info}{Deklerativ (z.B. Haskell)}
			Wir beschreiben, \textbf{was} berechnet werden soll.
		\end{info}
		\nextcolumn
		\begin{info}{Imperativ (z.B. Java)}
			Wir beschreiben, \textbf{wie} etwas berechnet werden soll.
		\end{info}
	\end{fancycolumns}
\end{frame}

\begin{frame}[fragile]{Haskell}
	\begin{fancycolumns}[T,widths={50,50}]
		\textbf{Deklerativ}\break\mbox{}\newline
		\begin{minipage}{0.5\textwidth}
			\begin{minted}[autogobble]{haskell}
				fib 0 = 0
				fib 1 = 1
				fib n = fib (n-1) + fib (n-2)
			\end{minted}
		\end{minipage}
		\nextcolumn	
		\textbf{Imperativ}\break\mbox{}\newline
	\begin{minipage}{0.5\textwidth}
		\begin{minted}[autogobble]{java}
			int firstNumber = 0;
			int secondNumber = 1;
			int fibonacci = 0;

			for (int i = 1; i < N; i++) {
				fibonacci = 
				firstNumber + secondNumber;
				firstNumber = secondNumber;
				secondNumber = fibonacci;
			}
		\end{minted}
	\end{minipage}
	\end{fancycolumns}
\end{frame}

\begin{frame}[t,fragile]{Haskell}
	\begin{enumerate}[$\rightarrow$]
		\item Haskell ist  \underline{Lazy}
		\item Umgang mit Unendlichen Listen
	\end{enumerate}
	
	\mbox{}\newline
	\mbox{}\newline

	\begin{center}
		\begin{minipage}{0.2\textwidth}
			\begin{minted}[autogobble]{haskell}
				take 5 [1..]
			\end{minted}
		\end{minipage}
		\end{center}


\end{frame}

\section{Blatt 1}
\subsection{Aufgabe 1a}
\begin{frame}{\insertsubsection} 
	\begin{exercise}{Ist der Algorithmus deterministisch?}
	Nein, der Algorithmus ist \textit{nicht deterministisch}, da in Schritt 1 eine zufällige
	initiale Gästeliste gezogen wird. Außerdem geben die Schritte 2 und 3 nicht an,
	für welche Gäste die Regeln zuerst angewandt werden sollen.
	\end{exercise}

	\mbox{}

	\begin{enumerate}[$\rightarrow$]
		\item "\textbf{Schritt 1} Zieh eine zufällige Liste von möglichen Gästen"
		\item Ein Algorithmus heißt \textbf{deterministisch}, wenn die Wirkung und die Reihenfolge der Einzelschritte eindeutig festgelegt ist. \footnote[1]{\url{https://www.dbs.ifi.lmu.de/Lehre/EIP/WS1415/skript/EiP-04-DatenAlgorithmen-Teil2.pdf}}
	\end{enumerate}  
\end{frame}

\subsection{Aufgabe 1b}
\begin{frame}{\insertsubsection} 
	\begin{exercise}{Terminiert der Algorithmus immer?}
		Nein, es ist möglich, dass der Algorithmus nicht terminiert, wenn die gezogenen
		initialen Listen immer zu einer ungenügend großen Gästeliste führen.
	\end{exercise}

	\mbox{}

	\begin{enumerate}[$\rightarrow$]
		\item \textbf{Terminierend}: Der Algorithmus endet für alle validen Schrittfolgen nach endlich vielen Schritten
	\end{enumerate}
\end{frame}

\subsection{Aufgabe 1c}
\begin{frame}{\insertsubsection} 
	\begin{exercise}{Ist der Algorithmus partiell korrekt?}
		Ja, denn Schritt 2 stellt sicher, dass alle Gäste hinzugefügt werden, sodass die
		Aussagen von Typ 1 erfüllt sind, und Schritt 3 entfernt alle Gäste, sodass Aussagen
		beider Typen erfüllt bleiben.
	\end{exercise}

	\mbox{}

	\begin{enumerate}[$\rightarrow$]
		\item Ein Algorithmus heißt \textbf{partiell korrekt}, wenn für alle gültigen Eingaben (Vorbedingung) das Resultat der Spezifikation (Nachbedingung) des Algorithmus entspricht. \footnote[1]{\url{https://www.dbs.ifi.lmu.de/Lehre/EIP/WS1415/skript/EiP-04-DatenAlgorithmen-Teil2.pdf}}
	\end{enumerate}
\end{frame}


\subsection{Aufgabe 2a}
\begin{frame}{\insertsubsection} 
	\begin{exercise}{Ist der Algorithmus iterativ oder rekursiv beschrieben?}
		Rekursiv
	\end{exercise}

	\mbox{}

	$$
	foo(a,b) = \begin{cases}
		a 				& \text{wenn } a = b \\
		foo(a, b-a) 	& \text{wenn } a < b \\
		foo(b, a-b)		& \text{wenn } a > b
	\end{cases}
	$$

\end{frame}


\subsection{Aufgabe 2b}
\begin{frame}{\insertsubsection} 
	\begin{exercise}{Berechnen Sie das Ergebnis für die Eingabe...}
		\begin{flalign*}
			foo(12, 9) &=_{12>9} foo(9, 12 - 9) = foo(9, 3) \\
					   &=_{9>3}  foo(3, 9 - 3)  = foo(3, 6) \\
					   &=_{3<6}  foo(3, 6 - 3)  = foo(3, 3) \\
					   &=_{3=3}  3
		\end{flalign*}	
		\mbox{}
	\end{exercise}
\end{frame}

\subsection{Aufgabe 2c}
\begin{frame}{\insertsubsection} 
	\begin{exercise}{Beschreiben Sie in Ihren Worten, was der Algorithmus berechnet.}
		Der Algorithmus berechnet den größten gemeinsamen Teiler zweier Zahlen.
	\end{exercise}

	\mbox{}

	\begin{enumerate}[$\rightarrow$]
		\item Euklidischer Algorithmus
		\item Der Algorithmus nutzt aus, dass sich der größte gemeinsame Teiler zweier Zahlen nicht ändert, wenn man die kleinere von der größeren abzieht. \footnote[1]{\url{https://de.wikipedia.org/wiki/Euklidischer_Algorithmus}}
	\end{enumerate}
\end{frame}


\subsection{Aufgabe 3a}
\begin{frame}{\insertsubsection} 
	\begin{exercise}{Ist der Algorithmus iterativ oder rekursiv beschrieben?}
		Iterativ
	\end{exercise}

	\mbox{}

	\begin{enumerate}[$\rightarrow$]
		\item Schleife
		\item Keine Funktionen, die sich selbst aufrufen
	\end{enumerate}
\end{frame}

\subsection{Aufgabe 3b}
\begin{frame}[fragile]{\insertsubsection} 
	\begin{exercise}{Gegeben die Eingabe $n = 3$. Führen Sie den Algorithmus zeilenweise aus und geben Sie alle Zwischenzustände an. \underline{Geben Sie auch den Rückgabewert an.}}
		\begin{flalign*}			
					  \langle1 \mid \emptyset \rangle 
			& \mapsto \langle2 \mid n=3 \rangle 
			  \mapsto \langle3 \mid n=3, a=1 \rangle \\
			& \mapsto \langle4 \mid n=3, a=1, b=1 \rangle 
			  \mapsto \langle5 \mid n=3, a=1, b=1 \rangle \\
			& \mapsto \langle6 \mid n=3, a=1, b=1, a_{alt}=1\rangle 
			  \mapsto \langle7 \mid n=3, a=1, b=1, a_{alt}=1 \rangle \\
			& \mapsto \langle8 \mid n=3, a=1, b=2, a_{alt}=1 \rangle 
			  \mapsto \langle4 \mid n=2, a=1, b=2, a_{alt}=1 \rangle \\
			& \dots \\
			& \mapsto  \langle9 \mid n=0, a=3, b=5, a_{alt}=2 \rangle \mapsto 3
		\end{flalign*}
	\end{exercise}

	\begin{enumerate}[$\rightarrow$]
		\item Aufgabe komplett Lesen: \underline{Geben Sie auch den Rückgabewert an.}
	\end{enumerate}
\end{frame}


\subsection{Aufgabe 3c}
\begin{frame}[fragile]{\insertsubsection} 
	\begin{exercise}{Beschreiben Sie in Ihren Worten, was der Algorithmus berechnet.}
		Der Algorithmus berechnet eine Folge von $n$ Zahlen, beginnend mit $1, 1,$ bei der
		ein Element jeweils die Summe der beiden Element davor ist. Diese Folge wird
		auch die Fibonacci-Folge genannt.
	\end{exercise}
\end{frame}


\section{Blatt 2}
\begin{frame}[fragile]{Funktionsdekleration} 
	\begin{minted}[autogobble,escapeinside=||]{haskell}
		foo :: Int -> Double -> Double
	\end{minted}

	\pause
	\vspace{-1.5em}
	\hspace{1.2cm}
	$\underbrace{\hspace{2.4cm}}_{\text{Parameter}}$
	
	\pause
	\vspace{-2.3em}
	\hspace{4.2cm}
	$\underbrace{\hspace{1.1cm}}_{\text{Return Type}}$
	
	\pause
	\vspace{1.0cm}
	\begin{minted}[autogobble,escapeinside=||]{haskell}
		foo x y = fromIntegral x * y
	\end{minted}

\end{frame}

\begin{frame}[fragile,t]{Funktionsdekleration} 
	\underline{Beispiel}: Wir wollen nur Tupel vom Typ \mintinline{haskell}{(Int, Int)} erlauben

	\begin{minted}[autogobble]{haskell}
		fst :: (Int, Int) -> Int 
	\end{minted}

	\begin{minted}[autogobble]{haskell}
		fst (a, b) = a 
	\end{minted}

\end{frame}

\begin{frame}[fragile,t]{Tupel (im Vergleich zur Liste)}
	\mintinline{haskell}{(1, 5, "Hallo")} \onslide<2->{\mintinline{haskell}{:: (Num, Num, String)}}


	\vspace{1cm}

	\begin{itemize}
		\item Fixe Anzahl an Elementen
		\item Verschiedene Typen erlaubt
	\end{itemize}
\end{frame}

\begin{frame}[fragile,t]{Tupel}
	\begin{minted}[autogobble]{haskell}
		fst (x, y) = x -- Erstes Element
		snd (x, y) = y -- Zweites Element  
	\end{minted}
\end{frame}

\begin{frame}[fragile]{Listen} 
	\begin{minted}[autogobble]{haskell}
		1 : [2,3]          = [1,2,3]       -- Element einfügen
		[1,2,3] ++ [4,5,6] = [1,2,3,4,5,6] -- Listen konkatenieren
		head [1,2,3]       = 1             -- 1. Element
		last [1,2,3]       = 3             -- Letzte ELement
		tail [1,2,3]       = [2,3]         -- Alle außer das 1.
		init [1,2,3]       = [1,2]         -- Alle außer das letzte
	\end{minted}
\end{frame}

\begin{frame}[fragile]{Typevariablen}
	\begin{minted}[autogobble]{haskell}
		fst :: (a, b) -> a
	\end{minted}
\end{frame}

\begin{frame}[fragile]{Typevariablen mit Klassen Constraints}
	\begin{minted}[autogobble]{haskell}
		(==) :: (Eq a) => a -> a -> Bool
	\end{minted}
\end{frame}

\begin{frame}[fragile]{ghci} 
	\begin{minted}[autogobble]{haskell}
		:t <Identifier>  -- Typ Signatur ausgeben lassen
		:l <filename.hs> -- Datei laden
		:r               -- Datei neu laden
	\end{minted}
\end{frame}

\begin{frame}{Ende}
	\center Danke für Eure Aufmerksamkeit
\end{frame}

\end{document}
